\PassOptionsToPackage{unicode=true}{hyperref} % options for packages loaded elsewhere
\PassOptionsToPackage{hyphens}{url}
%
\documentclass[letterpaper,]{article}
\usepackage{lmodern}
\usepackage{amssymb,amsmath}
\usepackage{ifxetex,ifluatex}
\usepackage{fixltx2e} % provides \textsubscript
\ifnum 0\ifxetex 1\fi\ifluatex 1\fi=0 % if pdftex
  \usepackage[T1]{fontenc}
  \usepackage[utf8]{inputenc}
  \usepackage{textcomp} % provides euro and other symbols
\else % if luatex or xelatex
  \usepackage{unicode-math}
  \defaultfontfeatures{Ligatures=TeX,Scale=MatchLowercase}
\fi
% use upquote if available, for straight quotes in verbatim environments
\IfFileExists{upquote.sty}{\usepackage{upquote}}{}
% use microtype if available
\IfFileExists{microtype.sty}{%
\usepackage[]{microtype}
\UseMicrotypeSet[protrusion]{basicmath} % disable protrusion for tt fonts
}{}
\IfFileExists{parskip.sty}{%
\usepackage{parskip}
}{% else
\setlength{\parindent}{0pt}
\setlength{\parskip}{6pt plus 2pt minus 1pt}
}
\usepackage{hyperref}
\hypersetup{
            pdfborder={0 0 0},
            breaklinks=true}
\urlstyle{same}  % don't use monospace font for urls
\usepackage[margin=1in]{geometry}
\setlength{\emergencystretch}{3em}  % prevent overfull lines
\providecommand{\tightlist}{%
  \setlength{\itemsep}{0pt}\setlength{\parskip}{0pt}}
\setcounter{secnumdepth}{0}
% Redefines (sub)paragraphs to behave more like sections
\ifx\paragraph\undefined\else
\let\oldparagraph\paragraph
\renewcommand{\paragraph}[1]{\oldparagraph{#1}\mbox{}}
\fi
\ifx\subparagraph\undefined\else
\let\oldsubparagraph\subparagraph
\renewcommand{\subparagraph}[1]{\oldsubparagraph{#1}\mbox{}}
\fi

% set default figure placement to htbp
\makeatletter
\def\fps@figure{htbp}
\makeatother


\date{}

\begin{document}

\newcommand{\pref}{\mathrm{F_{PRE}}}\newcommand{\postf}{\mathrm{F_{POST}}}\newcommand{\transf}{\mathrm{F_{TRAN}}}\newcommand{\invf}{\mathrm{F_{INV}}}\newcommand{\env}{\mathrm{Env}}

\hypertarget{problem-description}{%
\section{Problem Description}\label{problem-description}}

Given a set of integer variables \(x_1, \dots, x_n\) and their primed
versions \(x_1', \dots, x_n'\), it is possbile to describe a loop which
uses these variables by 3 boolean functions:
\[\mathrm{F_{PRE}}(x_1, \dots, x_n)\]
\[\mathrm{F_{TRAN}}(x_1,\dots,x_n,x_1',\dots,x_n')\]
\[\mathrm{F_{POST}}(x_1,\dots,x_n)\]

With:

\begin{description}
\tightlist
\item[\(\mathrm{F_{PRE}}(x_1, \dots, x_n) = 1\)]
denoting that the set of variables are a valid set of input variables to
the loop;
\item[\(\mathrm{F_{TRAN}}(x_1, \dots, x_n, x_1', \dots, x_n') = 1\)]
denoting that the unprimed set of input variables become its
corresponding primed versions after one interation of loop
\item[\(\mathrm{F_{POST}}(x_1, \dots, x_n) = 1\)]
denoting that the set of input variables are a valid set of ending state
of the loop
\end{description}

The problem is to find the invariant of the loop, a boolean function
\[\mathrm{F_{INV}}(x_1,\dots,x_n)\] such that
\[\begin{split}\mathrm{F_{PRE}}(x_1,\dots,x_n) & \Rightarrow \mathrm{F_{INV}}(x_1, \dots, x_n) \\
\mathrm{F_{INV}}(x_1,\dots,x_n) \land \mathrm{F_{TRAN}}(x_1,\dots,x_n,x_1',\dots,x_n') & \Rightarrow \mathrm{F_{INV}}(x_1',\dots,x_n') \\
\mathrm{F_{INV}}(x_1,\dots,x_n) &\Rightarrow\mathrm{F_{POST}}(x_1,\dots,x_n)\end{split}\]

\hypertarget{general-terminology}{%
\section{General Terminology}\label{general-terminology}}

\begin{description}
\item[Set]
A set, referred to in our context, is a set of points of \(Z^n\) space.
It could be expressed as a set of expressions on the \(n\)
variables.\[f_i(x_1, x_2, \dots, x_n) \ge 0\] It represents a geometric
region in \(Z^n\) space. For CLIA problems, all set involved in the
problem shall have linear boundaries. If the set of constraints of a set
are all of linear forms, in other words, if \(f_i\)'s are all linear
functions, then we denote such a set as a \textbf{linearly bounded set}.
\item[Finite Set]
If a set contains finite numbers of points from \(Z^n\) space, then we
call it a finite set. In other words, if \[\exists P\in Z^n, P \in R \]
has finite solutions, then \(R\) is a finite set.
\item[Transformation]
A transformation is a set of constraints set on
\((x_1, x_2, \dots, x_n)\) (input variables) and
\((x_1', x_2', \dots, x_n')\) (output variables) So that in one
interation of the loop, given values of all variables in last interation
as input variables, constraints on the new values of all variables could
be determined as output variables
\item[Deterministic Transformation]
A deterministic transformation is a transformation which could determine
a set of values of the output variables with the values of input
variables given. In other words, for a deterministic transformation
\(T\), \[\exists \{x_i'\}, T(\{x_i\}, \{x_i'\}) \] has and only has one
set of \(\{x_i'\}\) as its solution. When \(T\) is a deterministic
transformation, we denote that \[T(\{x_i\}) = \{x_i'\}\] or
\[T(P) = P' \] Given that \[P = \{x_i\}, P' = \{x_i'\}\]
\item[Linear Deterministic Transformation]
If a transformation \(T\) satisfies the following condition:
\[ T \Leftrightarrow (x_1' = x_1 + c_1) \land (x_2' = x_2 + c_2) \land
\dots \land (x_n' = x_n + c_n) \] with \(n\) input variables and \(n\)
output variables on \(Z^n\) space, then we call it a \textbf{Linear
Deterministic Transformation}. We also denote the n-dimensional vector
\[ \vec{c} =  (c_1, c_2, \dots c_n) \] as the \textbf{Transform Vector}
of the transformation.

Using the notation of deterministic transformation above, this
transformation could be expressed as
\[T(\{x_1, x_2, \dots, x_n\}) = \{x_1 + c_1, x_2 + c_2, \dots, x_n + c_n\} \]
\item[Envelope]
Given an input set \(R\) and a deterministic transformation \(T\).
Define the envelope of \(R\) under \(T\), \(\mathrm{Env}(R, T)\) as:
\[\mathrm{Env}(R,T) = R \lor \{P'| P' = T(P), P \in R\} \] We denote the
transformation from a set to its envelope under certain deterministic
transformation as one expansion of that set under such transformation,
and we could keep doing that expansion on the expanded envelope.
\item[Regions of a transformation]
If a transformation has certain conditions on its input variables that
is a set: \[ T \Leftrightarrow  R(\{x_k\})\land T'(\{x_k, x_k'\})\]
denote that set as the \textbf{Region} of the transformation. If a
transformation is in the form of a disjunction, with each part being a
region transformation without any overlaying parts:
\[\begin{split} T  \Leftrightarrow  &(R_1(\{x_k\})\land T_1'(\{x_k, x_k'\}))\\
&\lor(R_2(\{x_k\})\land T_2'(\{x_k, x_k'\}))\\
&\dots
\end{split} \] \[R_i \land R_j = \emptyset, \text{for any } i \neq j\]
Then denote this transformation as a \textbf{multi-region}
transformation, still denote each of these regions as \textbf{regions}
of the transformation and denote the transformations in these regions as
\textbf{subtransformations}.
\item[Extended Transformation]
If a multi-region transformation's regions do not cover the whole
\(Z^n\) space, then for the uncovered part of the space, or for the
\textbf{undefined regions} of the transformation, constraints on output
variables does not exist, so principally they could be any value. An
\textbf{extended transformation} could be introduced, for which in the
undfined regions of the original multi-region transformation,
\(x_k' = x_k\) is enforced for all variable pairs. We still call that
domain as undefined regions for convenience. Observe that this enforces
the subtransformation to be linear deterministic for the undefined
regions.
\item[Contain]
Given a set \(R\) and a transformation \(T\). If
\[\forall P \in R, T(P) \in R \] Then we note that \(T\) is
\textbf{contained} by \(R\), this could also be expressed as:
\[T \land R \Rightarrow R(\{x_i'\} \to \{x_i\}) \]
\end{description}

\hypertarget{equivalent-expression-for-the-problem}{%
\section{Equivalent expression for the
problem}\label{equivalent-expression-for-the-problem}}

Using the terminology defined above, certain observations could be
obtained for the problem:

\begin{itemize}
\tightlist
\item
  \(\mathrm{F_{PRE}}, \mathrm{F_{POST}}, \mathrm{F_{INV}}\) denotes
  three sets in \(Z^n\) space
\item
  \(\mathrm{F_{PRE}}\) should be a subset of \(\mathrm{F_{INV}}\)
\item
  \(\mathrm{F_{INV}}\) should be a subset of \(\mathrm{F_{POST}}\)
\item
  Transform denoted by \(\mathrm{F_{TRAN}}\) should be contained in
  \(\mathrm{F_{INV}}\)
\end{itemize}

These observations could serve as a set of equivalent expressions to the
original invariant synthesis problem.

\hypertarget{proof-on-finite-bound-problems}{%
\section{Proof on finite bound
problems}\label{proof-on-finite-bound-problems}}

\hypertarget{assumption-and-argument}{%
\subsection{Assumption and argument}\label{assumption-and-argument}}

Assume that the form of \(\mathrm{F_{PRE}}\) is
\[\mathrm{F_{PRE}}\Leftrightarrow C_1^{(0)}\land C_2^{(0)}\land\dots\]

And the form of \(\mathrm{F_{TRAN}}\) is: \[\begin{split}
\mathrm{F_{TRAN}}\Leftrightarrow ( &C_1^{(1)} \land C_2^{(1)} \land \dots
\\ &T^{(1)}_1 \land T^{(1)}_2 \land \dots \land T^{(1)}_n) \lor \\
( &C_1^{(2)} \land C_2^{(2)} \land \dots
\\ &T^{(2)}_1 \land T^{(2)}_2 \land \dots \land T^{(2)}_n) \lor \\
& \dots
\end{split}\] with elements, or atoms \(C_j^{(i)}\) having the form
\[C_j^{(i)} \Leftrightarrow  \begin{cases}x_k \ge c   \\ x_k \le c \\ x_l - x_k \ge c \end{cases}\]

And atoms \(T_j^{(i)}\) have the form
\[T_j^{(i)} \Leftrightarrow x_j' = x_j + c_j \]

We would like to prove the argument that given this assumption as a
constraint, if there exists an invariant that is a finite set, there
must exist another invariant that have the form
\[\mathrm{F_{INV}}\Leftrightarrow C_1 \land C_2 \dots \] with \(C_i\)
being in the same form as \(C_j^{(i)}\), and such invariant could be
obtained by expanding \(\mathrm{F_{PRE}}\) for finite times.

\hypertarget{terminology}{%
\subsection{Terminology}\label{terminology}}

\begin{description}
\tightlist
\item[Octagon]
An Octagon is a set with the form: \[ C_1 \land C_2 \land \dots\] And
with atomic constraints in the form being: \[ C_i \Leftrightarrow
\begin{cases}x_k \ge c   \\ x_k \le c \\ x_l - x_k \ge c \end{cases}\]
\item[Combination of Octagons]
A combination of octagons is a set with the form:
\[R_1 \lor R_2 \lor \dots \] And with each of these \(R_i\) being an
octagon.
\item[Octagonal Multi-region Transformation]
If the regions of a multi-region transformation are octagons, we denote
that transformation as a octagonal multi-region transformation.
\item[Linear Deterministic Octagonal Multi-region Transformation]
If for each regions of a octagonal multi-region transformation, the
subtransformations are linear Deterministic transformations, we denote
the whole transformation as a linear deterministic octagonal
multi-region transformation. \#\# Lemmas
\end{description}

Certain lemmas could be obtained to help the proof of the argument as
well.

\begin{description}
\tightlist
\item[Lemma 1]
A set is a combination of octagons iff. the equations of the boundaries
of this set is of form \(f_i\), with \(f_i\) being:
\[ f_i \Leftrightarrow \begin{cases} x_i = c_i \\ x_i - x_j = c_{ij}\end{cases}\]
\end{description}

Proof.

If set \(R\) is a combination of octagons, then the atomic expression of
its logical form is \(C_j^{(i)}\), so the boundary equation forms of
\(R\) shall be the equation version of the inequalities in
\(C_j^{(i)}\), which is of form \(f_i\)

If set \(R\) has boundary equations in form of \(f_i\), the logical form
of \(R\) shall consist of inequality version forms of these equations as
its atomic expression, which is \(C_j^{(i)}\), and we could always
convert this form into a disjunction normal form (DNF), which is the
same notation as it is in the definition of combinations of octagons,
thus \(R\) is a combination of octagons.

\begin{description}
\tightlist
\item[Lemma 2]
Transformation \(T\) is contained by set \(R\) iff. the envelope of
\(R\) under \(T\) should be \(R\) iff.
\(\{P'| T(P) = P', P \in R\} \subseteq R\)
\end{description}

Proof.

\[\forall P\in R, T(P) \in R \Leftrightarrow  \{P'| T(P) = P', P \in R\}\subseteq R \Leftrightarrow  R \lor \{P'| T(P) = P', P \in R\} = R\]

\begin{description}
\tightlist
\item[Lemma 3]
Conjunctions and disjunctions of combinations of octagons are still
combinations of octagons.
\end{description}

Proof. Given combination of octagons:
\[R_1 = r_{11} \lor r_{12} \dots r_{1n}\]
\[R_2 = r_{21} \lor r_{22} \dots r_{2n}\] \(R_1 \lor R_2\) is in the
same form of combinations of octagons definition, so it is a combination
of octagons as well \(R_1 \land R_2\) could be converted into
disjunction normal form, which is in the same form of combinations of
octagons definition as well, so it is a combination of octagons too.

\begin{description}
\tightlist
\item[Lemma 4]
The envelope of a combination of octagons \(R\) under a linear
deterministic octagonal multi-region transformation \(T\),
\(\mathrm{Env}(R,T)\), is still a combination of octagons
\end{description}

Proof.

Consider different portions in \(R\) that contributes to
\[R' = \{P'| P' = T(P), P \in R\}\] As the whole \(Z^n\) space is
divided by \(T\)'s regions without any overlaying parts, it should be
possible to split R into subsets without overlaying parts with each
subset being in a different region of \(T\).

Formally, denote the regions of \(T\) as \(D_1, D_2, \dots D_m\), \(R\)
could be written as \[R = R_1 \lor R_2 \lor \dots R_m \] with
\[ R_i \Rightarrow D_i\] and \[ R_i \land R_j = \emptyset, i \neq j \]

Then, for each these subsets, points in it have a contribution in
\(R'\), denote them as \[R_i' = \{P'| P' = T(P), P \in R_i\}\]

\(R_i\) is \(R\) separated by region boundaries of \(T\), thus its
boundaries shall be either a boundary from \(R\)'s boundaries or a
boundary from \(T\)'s region boundaries. These boundaries all have the
forms of \(f_i\), thus according to \textbf{Lemma 1}, \(R_i\) is a
combination of octagons.

As \(R_i \Rightarrow D_i\), denote the transform vector in \(D_i\) as
\(\vec{t_i}\), all points in \(R_i\) are moved by vector \(\vec{t_i}\)
to form \(R_i'\), thus the boundaries are just moved by \(\vec{t_i}\) as
well, they're still in forms of \(f_i\), thus \(R_i'\)s are combinations
of octagons as well.

Then using \textbf{Lemma 3} \(R' = R_1' \lor R_2' \lor \dots \lor R_m'\)
is a combination of octagons as well.

Also then \(\mathrm{Env}(R, T) = R \lor R'\) is a combination of
octagons.

\begin{description}
\tightlist
\item[Lemma 5]
If transformation \(T\) is contained in \(R\), and \(R_s \subseteq R\),
then \(\mathrm{Env}(R_s, T) \subseteq R\)
\end{description}

Proof.

\(T\) is contained in \(R\), so \(\{P'|P' = T(P), P\in R\} \subseteq R\)
(\textbf{Lemma 2})

And \[R_s \subseteq R\] thus
\[\{P'|P' = T(P), P \in R_s\} \subseteq \{P'|P' = T(P), P \in R\} \subseteq R \]
thus \[ \mathrm{Env}(R_s, T) = R_s \lor
\{P'|P' = T(P), P \in R_s\} \subseteq R\]

\hypertarget{proof}{%
\subsection{Proof}\label{proof}}

\begin{description}
\item[Theorem 1]
For an invariant synthesis problem with \(\mathrm{F_{PRE}}\) in the form
of a combination of octagons, an arbitrary \(\mathrm{F_{POST}}\) that
makes sense, and a linear deterministic octagonal multi-region
transformation as its \(\mathrm{F_{TRAN}}\), if :

\begin{itemize}
\tightlist
\item
  there exisits an invariant \(\mathrm{F_{INV}}'\) for this problem,
  and,
\item
  the set of \(\mathrm{F_{INV}}'\) is a finite set
\end{itemize}

Then, there should exist another \(\mathrm{F_{INV}}\) for the this
problem, and \(\mathrm{F_{INV}}\) shall be a combination of octagons,
and such \(\mathrm{F_{INV}}\) could be obtained by expanding
\(\mathrm{F_{PRE}}\) for finite times.
\end{description}

Proof.

First, we take the pre-condition \(\mathrm{F_{PRE}}\) and expand it to
its envelope \(\mathrm{Env}(\mathrm{F_{PRE}}, T)\), it is obivous that
\(\mathrm{F_{PRE}}\subseteq \mathrm{Env}(\mathrm{F_{PRE}}, T)\)
according to the definition of envelopes.

We denote the number of points in set \(R\) as \(|R|\), then we have
\(|R| \le |\mathrm{Env}(R, T)|\) according to the definition of
envelopes.

Denote that
\[\mathrm{Env}^n(R, T) = \mathrm{Env}(\mathrm{Env}^{n-1}(R, T), T)\] and
\[\mathrm{Env}^0(R, T) = R\]

Assume that
\[\forall n \in Z, \mathrm{Env}^{n-1}(\mathrm{F_{PRE}}, T) \neq \mathrm{Env}^{n}(\mathrm{F_{PRE}}, T) \].

This leads to
\[|\mathrm{Env}^{n-1}(\mathrm{F_{PRE}}, T)| \neq |\mathrm{Env}^{n}(\mathrm{F_{PRE}}, T)| \]

We also have
\[|\mathrm{Env}^{n-1}(\mathrm{F_{PRE}}, T)| \le |\mathrm{Env}^{n}(\mathrm{F_{PRE}}, T)|\]

Thus
\[  |\mathrm{Env}^{n-1}(\mathrm{F_{PRE}}, T)| < |\mathrm{Env}^{n}(\mathrm{F_{PRE}}, T)| \]

We are on \(Z^n\) space, thus all these should be integer numbers, thus
\[  |\mathrm{Env}^{n-1}(\mathrm{F_{PRE}}, T)| + 1 \le |\mathrm{Env}^{n}(\mathrm{F_{PRE}}, T)| \]

Expanding the set to its envelope shall at least increase the set size
by 1.

On the other hand, according to the original problem, \(T\) shall be
contained in \(\mathrm{F_{INV}}'\), thus for any subset of \(T\), its
envelop under \(T\) shall still be a subset of \(T\) (\textbf{Lemma 5}).

So
\[\forall n \in Z, \mathrm{Env}^n(\mathrm{F_{PRE}}, T) \subseteq \mathrm{F_{INV}}' \]
so \[|\mathrm{Env}^n(\mathrm{F_{PRE}}, T)| \le |\mathrm{F_{INV}}'|\]

But as \(\mathrm{F_{INV}}'\) is finite, \(|\mathrm{F_{INV}}'|\) is
finite as well, and the set size increases at least one after expansion.
\[\exists m \in Z, \mathrm{Env}^m(\mathrm{F_{PRE}}, T) > |\mathrm{F_{INV}}'|\]

That raise a contradiction, so the assumption shall not hold, thus
\[\exists n \in Z, \mathrm{Env}^{n-1}(\mathrm{F_{PRE}}, T) = \mathrm{Env}^n(\mathrm{F_{PRE}}, T) \]

We denote \(\mathrm{F_{INV}}= \mathrm{Env}^{n-1}(\mathrm{F_{PRE}}, T)\),
thus \[\mathrm{F_{INV}}= \mathrm{Env}(\mathrm{F_{INV}}, T) \] According
to \textbf{Lemma 2}, this means \(T\) is contained by
\(\mathrm{F_{INV}}\) and thus \(\mathrm{F_{INV}}\) is a invariant
solution for the original problem.

According to \textbf{Lemma 4}, \(\mathrm{F_{INV}}\) is generated by
expanding \(\mathrm{F_{PRE}}\) for finite times, as \(\mathrm{F_{PRE}}\)
is a combination of octagons, \(\mathrm{F_{INV}}\) is a combination of
octagons as well.

Q.E.D.

\hypertarget{intermediate-ideas-stable-regions-and-safe-zones}{%
\section{Intermediate ideas: Stable Regions and Safe
Zones}\label{intermediate-ideas-stable-regions-and-safe-zones}}

\begin{itemize}
\tightlist
\item
  A region is stable if

  \begin{itemize}
  \tightlist
  \item
    A set expanding into such a region would either:

    \begin{itemize}
    \tightlist
    \item
      Be expanding infinitely in such region and covering all points in
      a direction.
    \item
      Reach a fixed point
    \end{itemize}
  \item
    And either way, it would never exit such region again.
  \item
    That property could be captured syntactically using constraints on
    the formations of the transformation regions and transformation
    vectors
  \end{itemize}
\item
  With stable regions in a transformation identified, we could define a
  safe zone for the transformation

  \begin{itemize}
  \tightlist
  \item
    A safe zone of a transforamtion is a set in the space such that it
    contains all points that, after a finite number of transforamtion
    steps, would step into a stable region
  \item
    This could be obtained by ``reverse expanding'' the stable regions.
  \end{itemize}
\item
  We can defined the compliment of a safe zone as the unsafe zone of the
  corresponding transformation
\item
  If the unsafe zone is finite, the strongest invariant shall be
  obtained by keep expanding from the pre condition set.

  \begin{itemize}
  \tightlist
  \item
    Expanding that way, a set would either

    \begin{itemize}
    \tightlist
    \item
      Expand into a stable region and yield a part of invariant in that
      region.
    \item
      Reach a fixed point after some expansions

      \begin{itemize}
      \tightlist
      \item
        This could be proved in the same manner as the finite invariant
        proof.
      \end{itemize}
    \end{itemize}
  \end{itemize}
\end{itemize}

\hypertarget{general-ideas-any-system-that-will-not-create-a-loop-should-work}{%
\section{General ideas: Any system that will not create a loop should
work}\label{general-ideas-any-system-that-will-not-create-a-loop-should-work}}

\hypertarget{terminology-1}{%
\subsection{Terminology}\label{terminology-1}}

\begin{description}
\item[Transformation Graph]
For a deterministic multi-region transformation, define two sets:
\[V = \{R_i | R_i \text{ is a region of transformation }T\}\]
\[E = \{ (R_i, R_j) | \exists P \in R_i, T(P) \in R_j \} \] Then
\((V,E)\) effectively defined a directed graph. We note this graph as
the \textbf{transformation graph} of transformation \(T\)
\item[Loop-free transformation]
For a linear deterministic multi-region transformation, if its
transformation graph is acyclic, then we denote this transformation as a
\textbf{loop-free transformation}
\item[Strong Invariant]
For a linear deterministic multi-region transformation \(T\) with
\(\mathrm{F_{PRE}}\), \(\mathrm{F_{POST}}\) and one of its invariant
\(\mathrm{F_{INV}}\), if in addition to the definition of invariants,
this invariant satisfies:
\[\forall P \in \mathrm{F_{INV}}, \exists P' \in \mathrm{F_{PRE}}, \exists n \in Z^{+}, T^{(n)}(P') = P\]
\item[Continuation]
For a deterministic multi-region transformation \(T\), and a set \(S\),
if there exists another set \(S'\), so that
\[\forall P' \in S', \exists P \in S, \exists n \ge 0, T^{(n)}(P) = P'\]
and
\[\forall n \ge 0, \forall P \in S, \exists P' \in S', T^{(n)}(P) = P'\]
then we say that \(S'\) is a \textbf{continuation} of \(S\) under \(T\),
denoted as \[\mathrm{Cont}(S, T) = S'\]

Note: as when \(n = 0\), we have \(T^{(0)}(P) = P\), it is obvious that
a continuation of a set shall contain all the points in the original
set, in other words, \(S \subseteq \mathrm{Cont}(S, T)\)
\item[Modular Set]
For a set \(S\), we call the set \[S' = S \land (x_i = r_i \mod m_i)\]
as the \textbf{modular set} of \(S\) on \(x_i\) with \(r_i \mod m_i\) ,
denoted as \[S' = \mathrm{Modu}(S, x_i, r_i, m_i)\]
\item[Modular Partitions]
For a set S, we call the series of set
\[\mathrm{Modu}(S, x_i, 1, m_i), \mathrm{Modu}(S, x_i, 2, m_i), \dots, \mathrm{Modu}(S, x_i, m_i - 1, m_i)\]
as a series of \textbf{modular partition} of \(S\) on \(x_i\) with
\(m_i\)
\item[Modular Octagon]
We call the modular set of an octagon as a \textbf{modular octagon}
\item[Modular Partition of a Set]
For a linear deterministic multi-region transformation \(T\) with a set
\(S\), we call the following procedure of partitioning set \(S'\) as
determining the modular partition of \(S\) under \(T\)

First take partitions of \(S\) by the regions of \(T\):
\[ S_i = S \land R_i \] Then, for each \(S_i\), assume that the
transformation vector in \(R_i\) is
\[(\delta x_1^{(i)}, \dots, \delta x_n^{(i)}) \] we partition \(S_i\)
into \(\delta x_1^{(i)} \times \dots \times \delta x_n^{(i)}\) modular
partitions:
\[S_i^{(m_1, \dots, m_n)} = \mathrm{Modu}(\dots \mathrm{Modu}(\mathrm{Modu}( S, x_1, m_1, \delta x_1^{(i)}), x_2, m_2, \delta x_2^{(i)} ) \dots , x_n, m_n, \delta x_n^{(i)} ) \]
Expressed equivalently:
\[S_i^{(m_1, \dots, m_n)} = S \land (x_1 = m_1 \mod \delta x_1^{(i)}) \land \dots (x_n = m_n \mod \delta x_n^{(i)}) \]
And we denote this partition as:
\[S_i^{(m_1, \dots, m_n)} = \mathrm{MPart}(S, T, R_i, \vec{m}), \vec{m} = (m_1, \dots, m_n)\]
\item[Modular Expansion of a Octagon]
For a linear deterministic multi-region transformation \(T\) with a set
\(S\), we call the following procedure of determining set \(S'\) as
determining the modular expansion of \(S\) under \(T\)

First, take the modular partitions of \(S\) under \(T\), for a partition
\[S_i^{(m_1, \dots, m_n)} \] Take its envelope,
\[\mathrm{Env}(S_i^{(m_1, \dots, m_n)}) \] and taking the union of all
these envelopes as the output
\[S' = \bigvee_{i, m_1, \dots m_n} \mathrm{Env}(S_i^{(m_1, \dots, m_n)}) \]
Denoted \[S' = \mathrm{MExp}(S, T)\]
\item[Modular Extension of a Octagon]
For a linear deterministic octagonal multi-region transformation \(T\)
with a set \(S\), we call the following procedure of determining set
\(S'\) as determining the modular extension of \(S\) under \(T\)

First, take the modular partitions of \(S\) under \(T\), for a partition
\[S_i^{(m_1, \dots, m_n)} \] It is trival to note that this set shall be
a modular octagon, so it would be in the form of
\[C_1 \land C_2 \land \dots \land (x_1 = m_1 \mod \delta x_1^{(i)}) \land \dots (x_n = m_n \mod \delta x_n^{(i)})\]
Also note that \(R_i\) shall be an octagon, in form of
\[C_1'\land C_2' \land \dots \] We than do the following for this set:

\textbf{This definition of extension is currently broken, the operations
above could not guarantee that all extensible partitions are captured
and are extented in a correct way}

For each variable \(x_l\) in the problem, it has a corresponding
\(\delta x_l\) in region \(R_i\) of \(T\), for these values, if they all
satisfy the following properties: - \(\delta x_l = 0\), or -
\(\delta x_l > 0\), and \(R_i\) don't have forms of boundries like
\[ x_l \le c, x_l \pm x_m \le c\], or - \(\delta x_l < 0\), and \(R_i\)
don't have forms of boundries like \[ x_l \ge c, x_l \pm x_m \ge c\]

Then this partition would be locally extensible, we obtain the extension
by: - for any \(\delta x_l > 0\), remove all boundries in
\(S_i^{(m_1, \dots, m_n)}\) that are in the form
\[ x_l \le c, x_l \pm x_m \le c\], and - for any \(\delta x_l < 0\),
remove all boundries in \(S_i^{(m_1, \dots, m_n)}\) that are in the form
\[ x_l \ge c, x_l \pm x_m \ge c\], and

We denote the set that is obtained after doing these procedures above
for each variable as the local extension of \(S_i^{(m_1, \dots, m_n)}\)
\[\mathrm{LExt}(S_i^{(m_1, \dots, m_n)}, T, R_i)  \] The final output of
modular extension is obtained by taking the union of the local
extensions of all these partitions:
\[S' = \bigvee_{i, m_1, \dots m_n} \mathrm{LExt}(S_i^{(m_1, \dots, m_n)}, T, R_i)\]
Denoted \[S' = \mathrm{MExt}(S, T)\]
\end{description}

\hypertarget{assumption-and-argument-1}{%
\subsection{Assumption and Argument}\label{assumption-and-argument-1}}

Assume that the form of \(\mathrm{F_{PRE}}\) is
\[\mathrm{F_{PRE}}\Leftrightarrow C_1^{(0)}\land C_2^{(0)}\land\dots\]

And the form of \(\mathrm{F_{TRAN}}\) is: \[\begin{split}
\mathrm{F_{TRAN}}\Leftrightarrow ( &C_1^{(1)} \land C_2^{(1)} \land \dots
\\ &T^{(1)}_1 \land T^{(1)}_2 \land \dots \land T^{(1)}_n) \lor \\
( &C_1^{(2)} \land C_2^{(2)} \land \dots
\\ &T^{(2)}_1 \land T^{(2)}_2 \land \dots \land T^{(2)}_n) \lor \\
& \dots
\end{split}\] with elements, or atoms \(C_j^{(i)}\) having the form
\[C_j^{(i)} \Leftrightarrow  \begin{cases}x_k \ge c   \\ x_k \le c \\ x_l - x_k \ge c \end{cases}\]

And atoms \(T_j^{(i)}\) have the form
\[T_j^{(i)} \Leftrightarrow x_j' = x_j + c_j \]

The argument we would like to prove is 2-fold. Firstly, we would like to
prove that, if the transformation \(T\) is a loop-free, and we denote
the operator \(O\) as
\[O(S, T) = \mathrm{MExp}(\mathrm{MExt}(S, T), T) \] then there exists
\(n \in Z^{+}\), such that \[O^{n+1}(S, T) = O^{n}(S, T)\] Secondly, if
this \(O^n(S, T)\) satisfies \(O^n(S, T)\subseteq \mathrm{F_{POST}}\),
then this \(O^n(S, T)\) is a strong invariant of this problem.

\hypertarget{lemmas}{%
\subsection{Lemmas}\label{lemmas}}

\begin{description}
\tightlist
\item[Lemma 6]
For a deterministic multi-region transformation \(T\), and a set \(S\),
if \[ S = S_1 \lor S_2 \lor \dots \lor S_k \], and
\[S_i' = \mathrm{Cont}(S_i, T), i = 1, \dots, k\] then
\[S' = S_1' \lor S_2' \lor \dots \lor S_k' = \mathrm{Cont}(S, T) \]
\end{description}

Proof.

\(\forall P \in S'\), as \(S' = S_1' \lor \dots \lor S_k'\),
\[\exists S_i', P \in S_i'\]

And \(S_i' = \mathrm{Cont}(S_i, T)\), so
\[\exists P' \in S_i, \exists n \ge 0, T^{(n)}(P') = P\]

As \(S_i \subseteq S\), this proves
\[\exists P' \in S, \exists n \ge 0, T^{(n)}(P') = P\] and thus justify
the first part of continuation definition.

\(\forall P \in S, \forall n \ge 0\), as \(S = S_1 \lor \dots \lor S_k\)
\[\exists S_i, P \in S_i\]

And \(S_i' = \mathrm{Cont}(S_i, T)\), so
\[\exists P' \in S_i', T^{(n)}(P) = P'\]

As \(S_i' \subseteq S'\), this proves
\[\exists P' \in S', T^{(n)}(P) = P'\] and thus justify the second part
of continuation definition.

So that, \[S' = \mathrm{Cont}(S, T)\]

Q.E.D.

\begin{description}
\tightlist
\item[Lemma 7]
For a deterministic multi-region transformation \(T\) with
\(\mathrm{F_{PRE}}, \mathrm{F_{POST}}\) to form a problem, and a set
\(\mathrm{F_{INV}}\).
\[\mathrm{F_{INV}}\text{ is a strong invariant of the problem} \Leftrightarrow (\mathrm{F_{INV}}= \mathrm{Cont}(\mathrm{F_{PRE}}, T)) \land (\mathrm{F_{INV}}\subseteq \mathrm{F_{POST}})\]
\end{description}

Proof.

The forward direction:

If \(\mathrm{F_{INV}}\) is a strong invariant of the problem, then by
definition we have \(\mathrm{F_{INV}}\subseteq \mathrm{F_{POST}}\)

And
\[\forall P \in \mathrm{F_{INV}}, \exists P' \in \mathrm{F_{PRE}}, \exists n \in Z^{+}, T^{(n)}(P') = P\]
This proves the first part of continuation definition.

For the second part, consider \(\forall P \in \mathrm{F_{PRE}}\), assume
\(\exists n \ge 0\) \[T^{(n)}(P) \notin \mathrm{F_{INV}}\] As
\(P\in \mathrm{F_{PRE}}\subseteq \mathrm{F_{INV}}\) there must exist a
\(n\ge m\ge 0\), so that
\[T^{(m)}(P) \in \mathrm{F_{INV}}, T^{(m + 1)}(P) \notin \mathrm{F_{INV}}\]
This contradicts with the \(\mathrm{F_{INV}}\) property of
\[\forall Q \in \mathrm{F_{INV}}, T(Q) \in \mathrm{F_{INV}}\] So the
assumption should not hold, thus
\[\forall n \ge 0, T^{(n)}(P) \in \mathrm{F_{INV}}\] This proves the
second part of continuation definition.

So that the forward direction is proven.

Now the backward direction:

If \(S = \mathrm{Cont}(\mathrm{F_{PRE}}, T)\), according to the first
property of continuation, \(\forall P \in \mathrm{F_{INV}}\)
\[\exists n\ge 0, \exists Q \in \mathrm{F_{PRE}}, P = T^{(n)}(Q)\] then
according to the second property of continuation, we have
\[T(P) = T^{(n+1)}(Q) \in \mathrm{F_{INV}}\] This effectively proves
\[\forall P \in \mathrm{F_{INV}}, T(P) \in \mathrm{F_{INV}}\] Together
with \(\mathrm{F_{INV}}\subseteq \mathrm{F_{POST}}\) this proves
\(\mathrm{F_{INV}}\) is a invariant.

We already have the first property of continuation in the same form as
the strong invariant additional property, thus this proves
\(\mathrm{F_{INV}}\) is a strong invariant.

So that the backward direction is proven.

Q.E.D.

\begin{description}
\tightlist
\item[Lemma 8]
\textbf{This lemma is where I found the flaw, this lemma should be used
to prove that, a set is extensible ( S is extensible iff. all points in
S would not leave the region S is in after arbitrary number of
transformations ) iff. the boundary conditions in the ``Modular
Extension'' definition is met.}
\item[Lemma 9]
For a linear deterministic octagonal multi-region transformation \(T\)
with a octagon \(S\), if \(S\) is the subset of a region \(R_i\) of
\(T\), \(S \subseteq R_i\), and \(S\) satisfies:
\[\forall n \ge 0, \forall P \in S, T^{(n)}(P)\in R\] then we take the
modular extension of \(S\), \[S' = \mathrm{MExt}(S, T)\] \(S'\) shall be
a continuation of \(S\) under \(T\): \[S' = \mathrm{Cont}(S, T)\]
\end{description}

** As the flaw causes the definition of modular extension may need to be
modified, the proof of this lemma needs to be rewritten**

\begin{description}
\tightlist
\item[Lemma 10]
For a linear deterministic multi-region transformation \(T\) with a set
\(S\), \(S\) could be partitioned by regions of \(T\):
\[S_i = S \land R_i, i = 1, \dots, k\] \[S = S_1 \lor \dots \lor S_k\]
then modular expansion could be performed separately on these partions
then unioned:
\[\mathrm{MExp}(S,T) = \mathrm{MExp}(S_1, T) \lor \dots \lor \mathrm{MExp}(S_k, T) \]
\end{description}

Proof.

As the first step of modular expansion is to partition the input set,
this argument is trival to prove

\begin{description}
\tightlist
\item[Lemma 11]
For a linear deterministic octagonal multi-region transformation \(T\)
with a octagon \(S\), \(S\) could be partitioned by regions of \(T\):
\[S_i = S \land R_i, i = 1, \dots, k\] \[S = S_1 \lor \dots \lor S_k\]
then modular extension could be performed separately on these partions
then unioned:
\[\mathrm{MExt}(S,T) = \mathrm{MExt}(S_1, T) \lor \dots \lor \mathrm{MExt}(S_k, T) \]
\end{description}

Proof.

As the first step of modular extension is to partition the input set,
this argument is trival to prove

\textbf{the expression this lemma needs to be modified as modular
extension definition is to be modified, but the proof should remain
trival even after modification}

\hypertarget{proof-1}{%
\subsection{Proof}\label{proof-1}}

For an octagonal \(\mathrm{F_{PRE}}\) that is an octagon in a linear
deterministic octagonal multi-region loop-free transformation \(T\). As
the transformation graph for \(T\) is loop-free, performing
transformation continuously on any point in \(\mathrm{F_{PRE}}\) shall
end up in a region, in which any further transformation would not move
the point out of the region.

Thus, for any point in \(\mathrm{F_{PRE}}\) we could associate a finite
path of regions with it
\[(R_{p_1}, R_{p_2}, \dots R_{p_j}), p_i \neq p_j \text{ when } i\neq j \]
We could partition \(\mathrm{F_{PRE}}\) based on the different paths
associated with the points

\textbf{We need an additional lemma here to prove that the modular
extension operation is applicable to these partitions. I thought before
that being an octagon should guarantee that, and the extension operation
shall always preserve the octagon property so it is a trival argument,
but now the flaw is that the extension operation may not preserve that
property, so we need to modify both the extension operation's
applicability conditions and the operation itself, then provide a
concrete proof on this lemma}

When the modular expansion and modular extension operation is performed
on the whole set, take one partition with path
\((R_{p_1}, R_{p_2}, \dots, R_{p_j})\) to consider. Denote this
partition as \(S\)

According to the path definition, we should have
\(S \subseteq R_{p_1}\), and \[T^{(i)}(S) \subseteq R_{p_{i+1}}, i < j\]
\[T^{(i)}(S) \subseteq R_{p_j}, i \ge j\]

As \(p_i \neq p_j\) when \(i \neq j\), this means that
\[\forall i < j - 1, \forall P \in T^{(i)}(S), \exists l \ge 0, T^{(l)}(P) \in R_{p_j} \neq R_{p_{i + 1}}\]
So that according to lemma 8, \(T^{(i)}(S), i < j - 1\) would not
satisfy the boundary conditions of modular extension.

This means \(O^{(i)}(S), i < j - 1\) would not satisfy the boundary
conditions of modular extension, and when \(i < j - 1\)
\(O^{(i)}(S) = S \lor T(S) \lor \dots T^{(i)}(S)\)

Proof for this argument could be done by induction:

for \(i = 0\), \(O^{(i)}(S) = S = T^{(0)}(S)\) as \(0 < j - 1\), S would
not satisfy the conditions.

Assume \(O^{(k)}(S), k + 1 < j - 1\) does not satisfy the conditions,
and \(O^{(k)}(S) = S \lor T(S) \lor \dots T^{(k)}(S)\) then
\(O^{(k+1)}(S) = O(O^{(k)}(S)) = \mathrm{MExp}(\mathrm{MExt}(O^{(k)}(S)))\)
as \(O^{(k)}(S)\) does not satisfy the conditions,
\[\mathrm{MExt}(O^{(k)}(S)) = O^{(k)}(S) \] so
\[O^{(k+1)}(S) = \mathrm{MExp}(O^{(k)}(S)) = T(O^{(k)}(S)) \lor O^{(k)}(S)\]
put \(O^{(k)}(S)\) expression in, and we've got
\[O^{(k+1)}(S) = S \lor T(S) \lor \dots \lor T^{(k+1)}(S)\] And this
form shall not satisfy the conditions, either.

Thus by mathematical induction the original argument is proven

This means \(O\) operations do not apply \(\mathrm{MExt}\) on
\(O^{(i)}(S)\) when \(i < j - 1\)

When \(i = j\), \[O^{(i)}(S) = O^{(j)}(S) = O(O^{(j - 1)}(S))\], And
\[O^{(j-1)}(S) = O(O^{(j-2)}(S)) = \mathrm{MExp}(T\lor T(S) \lor \dots \lor T^{(j-2)}(S)) = O^{(j-2)}(S)\lor T^{(j-1)}(S)\]
According to the definition of paths,
\(\forall P \in T^{(j-1)}(S), \forall n \ge 0\)
\[T^{(n)}(P) \in R_{p_j}\] So according to lemma 9
\[\mathrm{MExt}(T^{(j-1)}(S)) = \mathrm{Cont}(T^{(j-1)}(S)) \] As
\[\mathrm{MExt}(O^{(j-2)}(S)) = O^{(j-2)}(S)\] and
\[T^{(j-1)}(S)\subseteq R_{q_j}, O^{(j-2)}(S) \land R_{q_j}  = \emptyset\]
if we partition \(O^{(j-2)}(S)\) by the regions to
\(P_1, P_2, \dots, P_t\) then \(P_1, P_2, \dots, P_t, T^{(j-1)}(S)\)
shall be a valid partition by regions of \(O^{(j-1)}(S)\)

Then according to lemma 11,
\(\mathrm{MExt}(O^{(j-1)}(S)) = \mathrm{Cont}(T^{(j-1)}(S))\lor O^{(j-2)}(S)\)

As \[O^{(j-2)}(S) = T\lor T(S) \lor \dots \lor T^{(j-2)}(S)\]
\(\mathrm{MExt}(O^{(j-1)}(S))\) shall satisfy the two properties of
continuation of \(S\) as well So
\[\mathrm{MExt}(O^{(j-1)}(S)) = \mathrm{Cont}(S)\] Then
\[ O^{(j)}(S) = \mathrm{MExp}(\mathrm{Cont}(S))\]

According to the definition of continuation, it is trival that
\[\mathrm{MExp}(\mathrm{Cont}(S)) = \mathrm{Cont}(S)\]

So \[O^{(j)}(S) = \mathrm{Cont}(S)\]

Also we have that \[O^{(i + 1)}(S) = O^{(i)}(S), i \ge j\]

For this particular partition, after \(j\)th \(O\) operation, we obtain
the continuation of the partition

Then for the entire \(\mathrm{F_{PRE}}\), after sufficient times of
\(O\) operations, every partition transforms to the continuation of that
partition

Thus according to lemma 6, the whole set of \(\mathrm{F_{PRE}}\) after
sufficient times of operations, turns into a continuation of
\(\mathrm{F_{PRE}}\)

According to lemma 7, this proves that, if this continuation \(C\) has
\(C\subseteq \mathrm{F_{POST}}\), then it is a strong invariant to the
problem.

\end{document}
